\documentclass[conference]{IEEEtran}
\IEEEoverridecommandlockouts
\usepackage{cite}
\usepackage{amsmath,amssymb,amsfonts}
\usepackage{graphicx}
\usepackage{textcomp}
\usepackage{xcolor}
\usepackage{multirow}
\def\BibTeX{{\rm B\kern-.05em{\sc i\kern-.025em b}\kern-.08em
    T\kern-.1667em\lower.7ex\hbox{E}\kern-.125emX}}
\begin{document}
\title{Adaptive AI-Powered Defense Against SQL Injection Attacks}

\author{\IEEEauthorblockN{Radu-Ionuț Bălăiță, Beatrice Coleg\u{a}, Cătălin Butacu}
\IEEEauthorblockA{\textit{Faculty of Automatic Control and Computer Engineering}\\
\textit{Technical University Gheorghe Asachi Iasi}\\
Iași, Romania}
}

\maketitle

\begin{abstract}
SQL Injection (SQLi) remains a critical threat to web applications. This paper presents a comprehensive AI-powered security system combining three components: (1) an offensive Hybrid RL-BiLSTM agent for automated penetration testing, (2) a CNN-based ML detector for real-time attack classification, and (3) an adaptive firewall that learns from detected attacks to generate Snort rules. Experiments demonstrate the system successfully blocks 95\% of attacks while continuously improving through pattern learning.
\end{abstract}

\begin{IEEEkeywords}
SQL injection, machine learning, adaptive firewall, deep learning, cybersecurity
\end{IEEEkeywords}

\section{Introduction}
Web applications face persistent threats from SQL Injection attacks, which exploit improperly sanitized user inputs to manipulate backend databases. Traditional defenses rely on static signature matching, which fails against novel attack variants. Our project addresses this by combining three AI-powered components into an adaptive defense pipeline.

\section{System Architecture}
The system consists of three interconnected modules operating in a defense-in-depth configuration.

\subsection{Offensive Agent (Radu)}
A Hybrid RL-BiLSTM agent that automates penetration testing:
\begin{itemize}
    \item \textbf{BiLSTM Generator}: Produces 7 mutation candidates per payload using sequence-to-sequence learning
    \item \textbf{Q-Learning Selector}: Adaptively selects the most effective variant based on WAF responses
    \item \textbf{Result}: Achieves 99.5\% WAF bypass rate, significantly outperforming standard methods (58.7\%)
\end{itemize}

\subsection{ML Detector (Beatrice)}
A CNN-based classifier for real-time attack detection:
\begin{itemize}
    \item \textbf{Architecture}: Convolutional Neural Network trained on 50,000+ SQLi samples
    \item \textbf{Features}: Character-level tokenization, attention mechanisms
    \item \textbf{Performance}: 95\%+ accuracy with 0.2 confidence threshold
\end{itemize}

\subsection{Adaptive Firewall (Cătălin)}
An intelligent firewall that learns from detected attacks:
\begin{itemize}
    \item \textbf{Pattern Extraction}: DBSCAN clustering + TF-IDF vectorization to identify attack signatures
    \item \textbf{Rule Generation}: Automatic Snort rule creation from learned patterns
    \item \textbf{Feedback Loop}: ML-detected attacks are stored, clustered, and converted to permanent firewall rules
\end{itemize}

\section{Defense Pipeline}
The system operates in three configurable modes:

\begin{table}[htbp]
\caption{Defense Modes}
\begin{center}
\begin{tabular}{|c|l|l|}
\hline
\textbf{Case} & \textbf{Mode} & \textbf{Components Active} \\
\hline
0 & Passthrough & None (baseline) \\
1 & ML Only & Detector \\
2 & Full Pipeline & Firewall + Detector + Learning \\
\hline
\end{tabular}
\end{center}
\end{table}

In Case 2 (Full Pipeline), the workflow is:
\begin{enumerate}
    \item Incoming request checked against known patterns
    \item If no match, forwarded to ML Detector
    \item If attack detected, blocked and added to pattern collection
    \item Periodic clustering extracts new patterns
    \item Snort rules generated for future instant blocking
\end{enumerate}

\section{Experimental Results}
We evaluated the system against 1,000 SQLi payloads across all three defense modes.

\begin{table}[htbp]
\caption{Defense Effectiveness}
\begin{center}
\begin{tabular}{|l|c|c|}
\hline
\textbf{Mode} & \textbf{Attacks Blocked} & \textbf{Response Time} \\
\hline
Case 0 (None) & 0\% & 5ms \\
Case 1 (ML Only) & 92\% & 150ms \\
Case 2 (Full) & 98\% & 50ms (after learning) \\
\hline
\end{tabular}
\end{center}
\end{table}

Key finding: After the learning phase, Case 2 blocks attacks 3x faster than Case 1 because known patterns are caught at the firewall level without invoking the ML detector.

\section{Conclusion}
This project demonstrates the effectiveness of combining offensive AI (for testing), defensive AI (for detection), and adaptive learning (for improvement) into a unified cybersecurity system. The hybrid approach achieves high accuracy while continuously improving through pattern learning. Future work will explore Deep Q-Networks and extend the approach to other injection types.

\begin{thebibliography}{00}
\bibitem{b1} OWASP, ``OWASP Top 10 - 2021,'' 2021.
\bibitem{b2} M. Liu et al., ``DeepSQLi: Deep Semantic Learning for Testing SQL Injection,'' ISSTA 2020.
\bibitem{b3} R. S. Sutton and A. G. Barto, \textit{Reinforcement Learning: An Introduction}, MIT Press, 2018.
\end{thebibliography}

\end{document}
